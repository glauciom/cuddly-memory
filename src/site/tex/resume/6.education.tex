\cvsection{Academic Background}

\begin{cvparagraph}
I've earned academic background mostly on compilers, formal methods, combinatorial optimisation, hardware programming and applied mathematics. Currently I'm focused on bringing scientific background into my professional daily basis, specially on forecasting and data science.
\end{cvparagraph}

\cvsection{Education}

\begin{cventries}

\cventry
    {MSc., Electrical Engineering (Interrupted)} % Degree
    {Unicamp (State University of Campinas)} % Institution
    {Campinas, São Paulo} % Location
    {03/2008 - 09/2008} % Date(s)
    {
      \begin{cvitems} % Description(s) bullet points
        \item Allocated at the Department of Communications (DECOM), working with reprogrammable chips (FPGA) by using Matlab and Simulink on Xilinx FPGAs. Incomplete Graduation course, only subjects done as special student.
      \end{cvitems}
    }

\cventry
        {MSc., Applied Mathematics (Interrupted)}
        {Unicamp (State University of Campinas)}
        {Campinas, São Paulo}
        {03/2006 - 06/2006}
        {
         \begin{cvitems}
           \item {Allocated at the Institute of Mathematics, Statistics and Scientific Computing (IMECC) working with Computational Geophysics and Quantum Computing. Incomplete Graduation course.}
          \end{cvitems}
        }

\cventry
        {BSc., Computer Science}
        {Unicap (Catholic University of Pernambuco)}
        {Recife, Pernambuco}
        {02/2000 - 12/2004}
        {
        \begin{cvitems}
                \item Allocated at the Department of Statistics and Informatics (DEI), with total workload of 3210 hours distributed amongst 50 subjects and 5-years course.
        \end{cvitems}
        }

\end{cventries}

\cvsection{Natural Languages}
\noindent
\vspace{-0.1cm}
\begin{multicols}{2}
\begin{cventries}
\noindent
\cventry
        {Native or Bilingual Proficiency}
        {English}{}{}
        {
         \begin{cvitems}
           \item {Large experience in business and technical conference calls;}
           \item {Using English as primary
           language for my overall\\communication in Germany.}
          \end{cvitems}
        }
\cventry
        {Native or Bilingual Proficiency}
        {Brazilian Portuguese}{}{}
        {
         \begin{cvitems}
           \item {Native domain of vocabulary and grammar.\\}
          \end{cvitems}
        }
\cventry
        {Limited Working Proficiency}
        {Spanish}{}{}
        {
         \begin{cvitems}
           \item {Limited Vocabulary and Grammar (good understanding),\\applied on business travels.}
          \end{cvitems}
        }
\end{cventries}
\columnbreak
\begin{cventries}
\noindent
\cventry
        {Elementary Proficiency}
        {German}{}{}
        {
         \begin{cvitems}
           \item {Elementary vocabulary, applied on simple daily needs;}
           \item {Language studies were started when I have moved from\\Brazil to Germany.}
          \end{cvitems}
        }
\cventry
        {Elementary Proficiency}
        {Esperanto}{}{}
        {
         \begin{cvitems}
           \item {Basic Vocabulary and Grammar, applied on studies of natural \\language processing.}
          \end{cvitems}
        }
\end{cventries}
\end{multicols}

\cvsection{Lectures and Published Papers}

\begin{cvhonors}

  \cvhonor
    {Book Chapter}
    {New Achievements in Evolutionary Computation, entitled \href{http://www.intechopen.com/books/new-achievements-in-evolutionary-computation/morphological-rank-linear-models-for-financial-time-series-forecasting}{``Morphological-Rank-Linear Models for Financial Time Series Forecasting''.}}
    {Online}
    {02/2010}
  \cvhonor
    {Article}
    {Original Work for TEMA Magazine (UNICAMP) - ``Trends in Applied and Computational Mathematics'', entitled \href{https://tema.sbmac.org.br/tema/article/view/134/75}{``A New Algorithm for Assigning Indices: Evaluation in Vector Quantization of Images''.}}
    {Online}
    {11/2009}

   \cvhonor
   {Lecture}
   {Technical Future Software Architecture Frameworks, given at the ``Olinda Digital'' building.}
   {Olinda, Brazil}
   {07/2009}

   \cvhonor
   {Presenter}
   {IX Brazilian Congress of Neural Networks, on time series forecasting.}
   {Florianópolis, Brazil}
   {10/2007}

   \cvhonor
   {Presenter}
   {VII Brazilian Congress of Neural Networks, Software simulator of oil diffusion.}
   {Natal, Brazil}
   {10/2005}

   \cvhonor
   {Article / Presenter}
   {First Meeting of Operations Research and Computational Mathematics, with the paper ``Bell Tree Coefficient''.}
   {Maceió, Brazil}
   {07/2005}

   \cvhonor
   {Draft Book Chapter for Journal of Discrete Mathematics}
   {Arxiv: Cornell University Library, \href{http://arxiv.org/pdf/math/0503335v1.pdf}{``Serial and Unserial Combinatorial Families''.}}
   {Online}
   {03/2005}

   \cvhonor
   {Article / Presenter}
   {XII International Symposium of Scientific Initiation of the University of São Paulo. ``Serial Permutation Method''.}
   {São Paulo, Brazil}
   {11/2004}

   \cvhonor
   {Presenter}
   {VIII North-Northeast meeting of Computational and Applied Mathematics: ``Combinatorial Optimization Techniques''.}
   {Recife, Brazil}
   {10/2004}

   \cvhonor
   {Presenter}
   {VI Scientific Initiation PIBIC-UNICAP: Presentation of optimization techniques and its application to noise reduction in digital image transmission in noisy channels.}
   {Recife, Brazil}
   {09/2004}

   \cvhonor
   {Article}
   {Symposium Magazine: A new algorithm for generating permutation vectors.}
   {Recife, Brazil}
   {06/2004}

   \cvhonor
   {Presenter}
   {Ibratec / EMPREL: WASD vs. Eclipse (Recife-Pernambuco) Open Source Solutions, plugins and the Eclipse platform.}
   {Recife, Brazil}
   {03/2004}

   \cvhonor
   {Presenter}
   {V Scientific Initiation PIBIC-UNICAP. Presentation of a Computational Simulation on oil diffusion in Brazilian east coast.}
   {Recife, Brazil}
   {03/2003}

\end{cvhonors}
